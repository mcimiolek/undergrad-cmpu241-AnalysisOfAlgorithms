\documentclass[12pt]{article}
\usepackage{fullpage}
\usepackage{latexsym}
\usepackage{color}
\usepackage{listings}
\usepackage{caption}
\usepackage{amsmath}
\usepackage{amsfonts}
\usepackage{etoolbox}   % for booleans and much more
\usepackage{verbatim}


%\pagestyle{empty}

% set up solutions environment
\newbool{hidesolutions}
\setbool{hidesolutions}{false}

% new environment
\newenvironment{solution}{}{}

% set conditional behaviour of environment
\ifbool{hidesolutions}{\AtBeginEnvironment{solution}{\comment}%f
\AtEndEnvironment{solution}{\endcomment}}{\AtBeginEnvironment{solution}{\begin{quote}}%
\AtEndEnvironment{solution}{\end{quote}}}


\lstnewenvironment{algorithm}[1][] %defines the algorithm listing environment
{   
    \lstset{ %this is the stype
        mathescape=true,
        frame=tB,
        numbers=left, 
        numberstyle=\normalsize,
        basicstyle=\normalsize, 
        keywordstyle=\color{black}\bfseries\em,
        keywords={input, output, return, datatype, function, in, if, else, for, while, begin, end, downto, to,}
        numbers=left,
        xleftmargin=.04\textwidth,
    }
}{}

\begin{document}

\begin{center}
{ \Large \bf CMPU-241 - Analysis of Algorithms (Fall 2018)}\\
  {\bf Rui Meireles \\
Homework \#1 -- \ifbool{hidesolutions}{Due September 13\textsuperscript{th}, 2018}{Matthew Imiolek}\\
 }
  \vspace{.15in}
% NAME:\underline{\ \ \ \ \ \ \ \ \ \ \ \ \ \ \ \ \ \ \ \ \ \ \ \ \ \ \ \ \ \ \ \ \ \ \ \ \ \ \ \ \ \ \ }\\

\end{center}




\begin{enumerate}
\item Consider the following algorithm:



\begin{tabular}{l l}

{\bf Input:} n-element array of numbers, A$[1\dots n]$\\
{\bf Output:} ?\\\\

\end{tabular}

\begin{algorithm}[]
Algo(A)
  foo = 0
  for i = 1 to A.length
    for j = A.length downto i+1
      if A[j] $\geq$ A[i]
          foo++
  return foo
\end{algorithm}


\begin{enumerate}
\item (20 points) Define the output of algorithm \texttt{Algo}, i.e. state what it represents.



  \begin{solution}   
  The output is a count of the number of cases where there was a value in a later index of an array $\geq$ the value in the current index of the array.
\end{solution}

\item Prove that \texttt{Algo} outputs what you claim it outputs:

\begin{enumerate}
\item (15 points) Define the loop invariant for the \texttt{for} loop of line 3.


  \begin{solution}
    Before each iteration of the line 3 loop, foo holds the total number of values in the subarray A[i..A.length] $\geq$ A[i-1] for each index A[1..i-1].
\end{solution}

\item (20 points) Use the loop invariant to prove correctness. Show that the invariant holds before the first iteration of the loop (initialization), between loop iterations (maintenance), and that it being true after the loop ends implies the algorithm's correctness (termination).

  \begin{solution}
    Initialization:
    
    1) Before 1st interation, foo = 0
    
    2) When i = 1, A[1..i-1].
    
    3) This cannot exist, therefore there cannot be a value later in the array $\geq$ it, as something cannot be $\geq$ something that doesn't exist, and thus the count should be 0, as foo is.

    
    Maintenance:
    
    1) Assume at the start if iteration i = k < n, foo holds the number of values in the subarray A[k..A.length] $\geq$ A[k-1] for each index A[1..k-1].
    
    2) We must now show that at the start of iteration i = k + 1, foo holds the number of values in the subarray A[k+1..A.length] $\geq$ A[k] for each index A[1..k].
    
    3) Durring iteration A[k] will be compared to each value in subarray A[k+1..A.length] to see if each value in subarray A[k+1..A.length] is $\geq$ than A[k]. (This is done in an if statement)
    4) When this is true, the value of foo is incremented 1.
    
    5) Given 4, at the start of iteration k + 1, foo holds the number of values in the subarray A[k+1..A.length] $\geq$ A[k] for each index A[1..k].
    

    Termination:
    
    1) Let n = A.length
    
    2) Loop ends when i = n + 1 because that is when the loop condition first fails
    
        3) If the loop ends when i = n + 1, by the loop invariant we know foo contains the count of all cases where there were values in a later index of A[1..n] $\geq$ than the value of the index in A[1..n] being examined.
\end{solution}

\end{enumerate}

\item (30 points) Use the instruction counting technique to calculate \texttt{Algo}'s worst-case running time as a function of input size, i.e. worst-case $T(n)$. Assume each instruction takes a constant amount of time $c_l$ to execute, where $l$ represents the line number. Also, recall this useful summation rule: $\displaystyle\sum_{k=1}^{n}k=\dfrac{n(n+1)}{2}$.

  \begin{solution}
    Worst Case:
    $ T(n)$ = c_1+c_2+c_3(n)+c_4$\displaystyle\sum_{i=1}^{n}i$+c_5$\displaystyle\sum_{i=1}^{n}(i-1)$+c_6$\displaystyle\sum_{i=1}^{n}(i-1)$+c_7

    c_1+c_2+c_3n+c_4($\dfrac{n(n+1)}{2}$)+c_5($\dfrac{n(n-1)}{2}$)+c_6($\dfrac{n(n-1)}{2}$)+c_7

    ($\dfrac{c_4}{2}$+$\dfrac{c_5}{2}$+$\dfrac{c_6}{2}$)n^2+(c_3+$\dfrac{c_4}{2}$+$\dfrac{c_5}{2}$-$\dfrac{c_6}{2}$)n+c_1+c_2+c_7

    an^2 + bn + c
\end{solution}

\item (15 points) Ignoring constant factors, are the best- and worst-case running times of \texttt{Algo} the same? I.e., does the running time depend on any input properties other than its size? Justify your answer by showing how the calculations from the previous question would or would not change for the best-case.

  \begin{solution}
    Yes, they are the same, which means that running time does not depend on any other input properties other than the array's size. Due to the nature of for loops, both will go through to completion everytime even if there are no greater or equal values later in the array from the index being compared to, and as going through the for loops takes up the most time, the loss of one instruction in the inner for loop in an best-case situation does not change calaculation enough to have a large input. The worst case shown above would become $\T(n)$ = c_1+c_2+c_3(n)+c_4$\displaystyle\sum_{i=1}^{n}i$+c_5$\displaystyle\sum_{i=1}^{n}(i-1)$+c_7 in the best case. This eventually becomes ($\dfrac{c_4}{2}$+$\dfrac{c_5}{2}$)n^2+(c_3+$\dfrac{c_4}{2}$+$\dfrac{c_5}{2}$)n+c_1+c_2+c_7, which still has a quadriatic term from having to go through the inner for loop and if statement everytime, even if it doesn't have to go through line 6 ever again since in best case the line 5 if statement always evaluates to false. As in both cases the run time is still dominated by that quadratic term the only input property that really matters is the array's size.
\end{solution}


\end{enumerate}
\end{enumerate}

\end{document}



