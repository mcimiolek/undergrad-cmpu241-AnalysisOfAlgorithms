\documentclass[12pt]{article}
\usepackage{fullpage}
\usepackage{latexsym}
\usepackage{color}
\usepackage{listings}
\usepackage{caption}
\usepackage{amsmath}
\usepackage{amsfonts}
\usepackage{graphicx}
\usepackage{etoolbox}   % for booleans and much more
\usepackage{verbatim} % for the comment environment
\usepackage{enumitem}

%\pagestyle{empty}

% set up solutions environment
\newbool{hidesolutions}
\setbool{hidesolutions}{false}

% new environment
\newenvironment{solution}{}

% set conditional behaviour of environment
\ifbool{hidesolutions}{\AtBeginEnvironment{solution}{\comment}
\AtEndEnvironment{solution}{\endcomment}}{\AtBeginEnvironment{solution}{\begin{quote}}
\AtEndEnvironment{solution}{\end{quote}}}



\lstnewenvironment{algorithm}[1][] %defines the algorithm listing environment
{   
    \lstset{ %this is the stype
        mathescape=true,
        frame=tB,
        numbers=left, 
        numberstyle=\normalsize,
        basicstyle=\normalsize, 
        keywordstyle=\color{black}\bfseries\em,
        keywords={input, output, return, datatype, function, in, if, else, for, while, begin, end}
        numbers=left,
        xleftmargin=.04\textwidth,
    }
}{}

\begin{document}

\begin{center}
{ \Large \bf CMPU-241 - Analysis of Algorithms (Fall 2018)}\\
  {\bf Rui Meireles \\
Homework \#3 -- \ifbool{hidesolutions}{Due October 9\textsuperscript{th}, 2018}{<Matthew Imiolek>}\\
 }
  \vspace{.15in}
% NAME:\underline{\ \ \ \ \ \ \ \ \ \ \ \ \ \ \ \ \ \ \ \ \ \ \ \ \ \ \ \ \ \ \ \ \ \ \ \ \ \ \ \ \ \ \ }\\

\end{center}


\begin{enumerate}

\item Use the master method to give a $\Theta$ asymptotic bound for the following recurrences. Or, if the master method can't be applied, please justify why.

\begin{enumerate}

\item (6 points) $T(n) = 2T(n/8) + \sqrt[3]{n}$

\begin{solution}
$f(n) = \sqrt[3]{n}$ \\
$a = 2$ \\
$b = 8$ \\
$n^{\log_{8}{(2)} = \sqrt[3]{n}$ \\
$f(n) = n^{\log_{b}{(a)}$ leading to case 2 being true. Therefore $T(n) = \theta{n^{\log_{8}{(2)}} lg n}$
\end{solution}

\item (6 points) $T(n) = T(\dfrac{n}{2}) + n - \cos n$

\begin{solution}
$f(n) = n-\cos(n)$ \\
$a = 1$ \\
$b = 2$ \\
$n^{\log_{2}{(1)} = 1$ \\
$f(n) \geq n^{\log_{b}{(a)}$ leading to case 3. However, the regularity condition doesn't hold making this a situation where the master theorem doesn't hold. af(n/b) \leq cf(n) when c is .01 and n is 11. $1(11/2-\cos(11/2) = 4.791 \geq 0.1(11-\cos(11) = 1.002$
\end{solution}

\item (6 points) $T(n) = \log_{16}{(2)} T(\dfrac{3}{5}n) + 1$

\begin{solution}
$f(n) = 1$ \\
$a = \log_{16}{(2)} = 1/4$ \\
$b = 5/3$ \\
$n^{\log_{5/3}{(1/4)}$ \\
As $ n^{\log_{b}{(a)}$ is a negative exponent the master theorem doesn't work.
\end{solution}


\item (6 points) $T(n) = 82T(\dfrac{n}{9}) + n^2 + n\lg{n}$

\begin{solution}
$f(n) = n^2 + n\lg{n}}$ \\
$a = 82$ \\
$b = 9$ \\
$n^{\log_{9}{(82)} = n^(2.0056)$ \\
$f(n) \leq n^{\log_{b}{(a)}$ leading to case 1 being true. Therefore $T(n) = \theta{n^{\log_{9}{(82)}}}$
\end{solution}


\item (6 points) $T(n) = T(\dfrac{n}{2}) - n^3 + n^2$

\begin{solution}
The master theorem doesn't work because f(n) is how long a combination should take, and this cannot be negative in a program and work in the master theorem.
\end{solution}


\end{enumerate}


\item (25 points) How many nodes of depth $d$ can a heap have? Provide a tight range and use mathematical induction to prove it. Hint: think about the influence of the tree's height on the number of nodes at a given depth.

\begin{solution}
A heap can have up to $2^d$ nodes at depth d and as few as 1 node. \\

Worst Case: \\

Guess Bound by Substitution: \\
T(0) = 1\\
T(1) = T(0)*2 = 2*2 = 4\\
T(2) = T(1)*2 = 4*2 = 8\\
T(3) = T(2)*2 = 8*2 = 16\\
T(d) = $2^d$\\
Base Case: \\
When the heap has a depth of 0. In this case the only node possible is the root, so there can only be 1 node. $2^0$ is 1, therefore $2^d$ works for the base case. \\
Inductive Step: \\
Assume that a tree of depth d has $2^d$ nodes. \\
At depth d+1 each node at depth d gains 2 children in the worst case. Thus at depth d+1 there are $2*2^d$ nodes as we are doubling the number of nodes from the previous level which we know is $d^2$, which is the same as $2^{d+1}$ nodes, showing a heap can have $2^d$ nodes at any depth d.

Best Case:\\

Guess Bound by Substitution: \\
T(0) = 1\\
T(1) = T(0)\\
T(2) = T(1)\\
T(3) = T(2)\\
T(d) = 1\\
Base Case: \\
When the heap has a depth of 0. In this case the only node possible is the root, so there can only be 1 node. 1 is 1, therefore 1 node works for the base case. \\
Inductive Step: \\
Assume that a tree of depth d has 1 nodes. \\
At depth d+1 each node at depth d gains 1 child. Thus at depth d+1 there is 1 node as we are giving the single node at depth d a single child otherwise it does not actually reach depth d+1, showing a heap must have at least 1 node at any depth d.
\end{solution}

\item
Recall the heapsort algorithm:

\begin{tabular}{ p{1.4cm} p{13.3cm} }
{\bf Input:} & an n-element array of numbers, A$[1\dots n]$; \\
{\bf Output:}  & a permutation of A, $\{A_1^`,A_2^`,\dots,A_n^`\}$, such that $A_1^` \leq A_2^`\leq \dots A_n^`$.\\\\
\end{tabular}

\begin{algorithm}[]
HeapSort(A)
  Build-Max-Heap(A)
  for i = A.length downto 2 do 
    swap A[1] and A[i]
    A.heap-size -= 1			
    Max-Heapify(A,1)
  \end{algorithm}

Assume that the \texttt{Build-Max-Heap} and \texttt{Max-Heapify} procedures are correct and have the semantics specified in class. Let's now prove the correctness of heapsort.

\begin{enumerate}

\item (15 points) State the loop invariant for the loop of line 3.

\begin{solution}
Just before the start of iteration i, A has 2 sub-arrays. A[1...i] is a max head of the i smallest values in A, and A[i+1...A.length] contains the remaining values in A sorted smallest to largest.
\end{solution}

\item (30 points) Use the loop invariant you defined to prove heapsort's correctness.

\begin{solution}
Initialization: \\
At the start of the first iteration Build-Max-Heap(A) has been called, and as we assume it works, A is a max heap containing all the values. As i = A.length for this first iteration, the sub-array A[1...i] is the same as the whole of A, which we know is a max heap of all the values, which therefore includes all the values. As there are no other values A[i+1...A.length] should be empty, which it is as no values have be designated as being there yet. \\

Maintenance: \\
Assume the invariant holds for iteration i. Thus, before iteration i+1, A[1..i] is a max heap of the i smallest values, and A[i+1...A.length] contains all the other values sorted smallest to largest. Swapping A[1] and A[i] puts the next largest value as the value before A[i+1...A.length], without messing up the order of values in that sub-array. By making the heap-size in A one less (basically decrementing i) the sub-array A[i+1...A.length] is now one larger, but due to switching A[1] and A[i] it still contains the A.length largest values ordered from least to greatest. A[1...i] still contains the i least values in A, and with the call to Max-Heapify they are once again a max heap. Therefore, the loop invariant is again the same for the next loop. \\

Termination: \\
When i = 1, A[1...i] has the lowest value in A, and A[i+1...A.length] has the remaining values from least the greatest. Therefore A has become completely least to greatest as a whole and the heap has been sorted.
\end{solution}

\end{enumerate}


\end{enumerate}

\end{document}



